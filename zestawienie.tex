\documentclass[withmarginpar]{book}
\usepackage{fontspec}
\usepackage{polyglossia}
% po co:?
% \usepackage{etoolbox}
\setmainlanguage{polish}
\setotherlanguage{english}
\usepackage{csquotes}

\usepackage{wrapfig}

% https://tex.stackexchange.com/questions/16765/biblatex-author-year-square-brackets
% \usepackage[style=authoryear,backend=biber,backref=true,urldate=short]{biblatex}
\usepackage[style=ext-authoryear,backend=biber,backref=true,urldate=short]{biblatex}
\DeclareOuterCiteDelims{parencite}{\bibopenbracket}{\bibclosebracket}
\addbibresource{all.bib}
\DefineBibliographyStrings{polish}{%
  byeditor = {red.},
  urlseen = {dostęp},
}

% http://tex.stackexchange.com/questions/166337/quotation-mark-quotation-sign-xelatex-polyglossia-csquotes
\DeclareQuoteStyle{polish}% I looked it up on Wikipedia, no idea if it's right
  {\quotedblbase}
  {\textquotedblright}
  [0.05em]
  {\textquoteleft}
  {\textquoteright}

% psuje tytuły???:
%  \usepackage{ulem}
\usepackage{metalogo}
%\usepackage[polish]{cleveref}
\def\eob{ę}
\usepackage{xcolor}


\setmainfont[Mapping=tex-text]{TeX Gyre Termes}

\usepackage{relsize}

% https://tex.stackexchange.com/questions/139459/vref-and-input-command/167539#167539
% nie działa!
\usepackage[polish]{varioref}
\vrefwarning
% dla varioref!
\usepackage{hyperref}
%\usepackage{cleveref}

\usepackage{graphicx}
% nieintuicyjne:
%\usepackage{rviewport}
% [hyphens]: options clash
\usepackage{url}
%\usepackage{natbib}

% program name
\newcommand{\pname}[1]{\textsf{#1}}


% file name
\newcommand{\fname}[1]{\texttt{#1}}

\newcommand{\uname}[1]{\texttt{'#1'}}
\newcommand{\ucode}[1]{\texttt{U+#1}}
\newcommand{\usi}[1]{\texttt{#1}}

% Aletheia
\newcommand{\aname}[1]{\texttt{#1}}
\newcommand{\acode}[1]{\texttt{#1}}

% MUFI
\newcommand{\mname}[1]{\texttt{'#1 \textsc{<mufi>'}}}
\newcommand{\mcode}[1]{\texttt{M+#1}}



\usepackage{draftwatermark}
\usepackage[doublespacing]{setspace}

% \usepackage[draft]{fixme}

% nie działa:?
%\renewcommand{\topfraction}{0.9}
\renewcommand{\floatpagefraction}{0.9}	% require fuller float pages
\renewcommand{\topfraction}{0.9}	% max fraction of floats at top
\setcounter{topnumber}{5}
\setcounter{totalnumber}{5}  

\usepackage{expex}

\lingset{glhangstyle=none}
\defineglwlevels{N,TUS}
%\defineglwlevels{Pol,Urb}
%\newcommand{\bg}{\begingl[everyglBandtkie={\color{blue}},everyglŁoś={\color{green}},everyglKucała={\color{violet}},%
\let\Index\relax

 \newcommand{\bg}{\begingl}




\begin{document}


\title{Polskie prawo autorskie\\Zestawienie ustaw}

\author{Janusz S. Bień}

\date{\today}


\maketitle


\section{Wstęp}
\label{sec:wstęp}

\begin{description}
\item[1935] OBWIESZCZENIE MINISTRA SPRAWIEDLIWOŚCI z dnia 25 kwietnia
  1935 r.  w sprawie ogłoszenia jednolitego tekstu ustawy z dnia 29
  marca 1926 r. o prawie autorskiem.

  Na podstawie art. 2 ustawy z dnia 22 marca 1935 r. o zmianie ustawy
  z dnia 29 marca 1926 r. o prawie autorskiem (Dz. U. R. P. Nr. 26,
  poz. 176) ogłaszam w załączniku do obwieszczenia niniejszego
  jednolity tekst ustawy z dnia 29 marca 1926 r. o prawie autorskiem
  (Dz. U. R. P. Nr. 48 poz. 286) z nową numeracją artykułów z
  uwzględnieniem zmian wprowadzonych przez rozporządzenie Prezydenta
  Rzeczypospolitej z dnia 11 kwietnia 1927 r. (Dz. U. R. P. Nr. 36,
  poz 318), przez ustawę z dnia 22 marca 1935 r. (Dz. U. R. P. Nr. 26,
  poz 176) oraz przez inne przepisy obowiązujące.

Minister Sprawiedliwości: Czesław Michałowski
%Załącznik do obwieszczenia Ministra Sprawiedliwości z dnia 25 kwietnia 1935 r. (poz. 260).

Publikator: Dziennik Ustaw Rzeczypospolitej Polskiej z 1935 r., nr 36, poz. 260.

\end{description}

\chapter{I. POSTANOWIENIA OGÓLNE.}
\label{cha:i.-post-ogoln}

\section{Art. 1.  Przedmiot prawa autorskiego.}
\label{sec:art.-1}

\subsection{1935}
\label{sec:art.-1-1}

Przedmiotem prawa autorskiego jest od chwili ustalenia w jakiej bądź
postaci (słowem żywem, pismem, drukiem, rysunkiem, barwą, bryłą,
dźwiękiem, mimiką, rytmiką) każdy przejaw działalności duchowej,
noszący cechę osobistej twórczości.

Należą tu w szczególności: 

Dzieła utrwalone słowem żywem, pismem,
drukiem; mowy, referaty, wykłady, kazania, improwizacje, pamiętniki;
wydane i niewydane książki, broszury, artykuły oraz przygotowawcze do
nich narzuty, plany, zarysy i szkice; cały obszar produkcji
literackiej, naukowej, a także praktycznej, o ile ostatnia posiada
ślady indywidualnego ujęcia treści.

Kompozycje muzyczne wszelkiego rodzaju, czy to samoistne, czy związane ze słowem.

Dzieła z zakresu wszelkich sztuk graficznych i plastycznych:
rysunkowe, malarskie, rytownicze, litograficzne, rzeźbiarskie,
grawerskie, architektoniczne, dzieła sztuki zdobniczej, stosowanej do
rzemiosł i przemysłu, bez względu na ich rodzaj, rozmiary i wartość
materjału, zdjęcia fotograficzne lub otrzymane w podobny do fotografji
sposób, ilustracje naukowe, mapy i inne pomoce naukowe. W tem
wszystkiem korzystają z ochrony zarówno wykonanie ostateczne, jak
prowadzące do niego szkice, rysunki, plany, modele, projekty.

Utwory sztuki mimicznej (pantomina), rytmicznej (choreografja), żywe
obrazy, produkcje kinematograficzne i tym podobne, utrwalone w
scenarjuszach, rysunkach, fotografjach lub choćby tylko w pamięci
pewnej liczby osób.

\section{Art. 2.  Przedmiot prawa autorskiego.}
\label{sec:art.-2}

\subsection{1935}
\label{sec:art.-2-1}


Opracowanie cudzych utworów, jak: tłumaczenie, przeróbka,
przystosowanie, transpozycja na inną technikę artystyczną, układ
muzyczny, przeróbka na muzyczne instrumenty mechaniczne, przeniesienie
na film kinematograficzny i t. p., są również przedmiotem prawa
autorskiego. Wykonywanie takiego prawa zależy od zezwolenia twórcy
oryginału (zależne prawo autorskie). Zezwolenie jest zbędne, gdy prawo
autorskie co do oryginału wygasło. Zezwolenie traci moc, jeżeli
opracowanie nie ukazało się w przeciągu lat pięciu.

Wymóg uzyskania zezwolenia nie stosuje się do dzieł, które mają cechy
samodzielnej twórczości, chociaż podnietę do nich dał utwór cudzy.

\section{Art. 3.  Przedmiot prawa autorskiego.}
\label{sec:art.-3}

\subsection{1935}
\label{sec:art.-3-1}

Prawo autorskie do utworów fotograficznych lub otrzymanych w podobny
do fotografji sposób istnieje pod warunkiem, że zastrzeżenie wyraźne
uwidoczniono na odbitkach. Na odbitkach fotograficznych i
reprodukcjach, otrzymywanych w podobny do fotografji sposób, na
filmach, a także na nutach dla mechanizmu, na walcach fonograficznych
i tym podobnych przyrządach, odtwarzających utwór w sposób
mechaniczny, należy uwidocznić rok zdjęcia lub przeniesienia. W braku
podania roku prawo autorskie do takich utworów wtedy tylko ma skutek
przeciw osobom trzecim, jeżeli wiedziały, że czas trwania prawa
autorskiego jeszcze nie upłynął.

\section{Art. 4. Przedmiot prawa autorskiego.}
\label{sec:art.-4}

\subsection{1935}
\label{sec:art.-4-1}

Przedmiotem prawa autorskiego nie są:

1) ustawy, rozporządzenia, orzeczenia sądów i innych władz, oraz pisma
i formularze urzędowe, przeznaczone przez władze do wiadomości
powszechnej;

2) proste informacje dziennikarskie (wiadomości bieżące, dział
drobnych wypadków i t. p.).

\section{Art. 5. Przedmiot prawa autorskiego.}
\label{sec:art.-5}

\subsection{1935}
\label{sec:art.-5-1}

Utwory, określone w art. 1 — 3 korzystają z ochrony od chwili
prawowitego ich ukazania się (wydania, wygłoszenia, wystawienia i
t. p.) w następujących wypadkach:

1) jeżeli ich twórcami są obywatele Państwa Polskiego lub w Polsce zamieszkali cudzoziemcy;

2) jeżeli ukazały się naprzód w Polsce lub równocześnie w Polsce i zagranicą;

3) jeżeli wydane zostały naprzód w języku polskim;

4) jeżeli ochrona wynika z układów międzypaństwowych lub uzasadnia ją wzajemność.

Względem utworów, które jeszcze się nie ukazały, ochrona służy wszystkim twórcom, obywatelom Państwa Polskiego i cudzoziemcom.

\section{Art.  6. Podmiot prawa autorskiego.}
\label{sec:art.-6}

\subsection{1935}
\label{sec:art.-6-1}

Prawo autorskie należy w zasadzie do twórcy dzieła.

W braku dowodu przeciwnego za twórcę poczytuje się osobę, której
nazwisko zaznaczono na dziele albo ogłoszono przy wykonaniu lub
wystawieniu utworu.

\section{Art.  7. Podmiot prawa autorskiego.}
\label{sec:art.-7}

\subsection{1935}
\label{sec:art.-7-1}

Wydawcom zbiorów pieśni ludowych, melodji, przysłów, bajek, powieści,
wzorów stylu budowlanego i innych utworów sztuki ludowej, wypisów,
antologji, starych rękopisów, edycji krytycznych służy prawo
autorskie, o ile opracowanie wydawnicze (wybór, układ, ustalenie
tekstu i t. p.) ma cechy twórczości (art. 1).

\section{Art.  8. Podmiot prawa autorskiego.}
\label{sec:art.-8}

\subsection{1935}
\label{sec:art.-8-1}

Prawo autorskie do dzieł zbiorowych (encyklopedji, roczników,
kalendarzy i t. p.) oraz do czasopism jest podwójne: do całości służy
wydawcy, do poszczególnych części ich twórcom. Współpracownicy dzieł
zbiorowych, jeżeli otrzymują honorarjum autorskie, nie mogą gdzie
indziej wydawać opracowanych przez siebie części przez lat trzy od
czasu ukazania się ich w dziele zbiorowem. Współpracownicy czasopism
mogą wydawać gdzie indziej swe prace po ukazaniu się ich w całości w
piśmie perjodycznem. To ograniczenie upada z chwilą, gdy w czasopiśmie
dalszy ciąg utworu nie ukazuje się bez winy twórcy dłużej niż przez
trzy miesiące.

Twórcy dzieł łączonych (np. opera i libretto, melodja i tekst, powieść
i ilustracja) mają wspólne prawo autorskie co do całości, jednak każdy
w swym zakresie zachowuje prawo odrębne.

Do dzieł nierozłącznych (np. powieść lub utwór dramatyczny, napisany
przez kilku autorów wspólnie) stosuje się odpowiednio przepisy o
współwłasności.

\section{Art.  9. Podmiot prawa autorskiego.}
\label{sec:art.-9}

\subsection{1935}
\label{sec:art.-9-1}

Twórcę dzieła, wydanego bez podania nazwiska (anonim), albo pod
nazwiskiem zmyślonem (pseudonim), zastępuje w obronie praw autorskich
wydawca, w braku wydawcy nakładca. Zastępstwo to rozciąga się także na
obronę praw osobistych. Zastępstwo ustaje, jeżeli twórca poda do
wiadomości publicznej swe nazwisko.

\section{Art.  10. Podmiot prawa autorskiego.}
\label{sec:art.-10}

\subsection{1935}
\label{sec:art.-10-1}

Prawo autorskie do utworów fotograficznych lub otrzymanych w podobny
do fotografji sposób, do filmów kinematograficznych i do przeróbki
utworów muzycznych na instrumenty muzyczne służy przedsiębiorcy, w
razie zaś zamówienia dzieła — zamawiającemu.

\section{Art.  11. Podmiot prawa autorskiego.}
\label{sec:art.-11}

\subsection{1935}
\label{sec:art.-11-1}

Stosunki, uregulowane w przepisach art. 6 — 10, można urządzić w
umowie inaczej.  Treść prawa autorskiego.


\section{Art.  12. Podmiot prawa autorskiego.}
\label{sec:art.-12}

\subsection{1935}
\label{sec:art.-121-1}

Twórca rozporządza swem dziełem wyłącznie i pod każdym względem; w
szczególności rozstrzyga czy dzieło ma się ukazać, czy ma być
odtworzone, rozpowszechnione i w jaki sposób.

Ochrona praw osobistych służy każdemu twórcy bez względu na istnienie
lub nieistnienie prawa autorskiego (art. 62).


\end{document}


Art. 7.


Art. 8.


Art. 9.


Art. 10.


Art. 11.

Art. 12.


R o z d z i a ł   II.
OGRANICZENIE PRAW AUTORSKICH.
Art. 13.

 W dziedzinie piśmiennictwa wolno każdemu, pod warunkami art. 17:
 1) przedrukowywać w prasie aktualne dyskusyjne artykuły prasowe na tematy ekonomiczne, polityczne lub religijne, jeżeli zostały one ogłoszone bez zastrzeżenia;
 2) przedrukowywać w czasopismach lub w dziełach, przeznaczonych na publikacje tego rodzaju, mowy, wygłoszone na zebraniach lub rozprawach o charakterze publicznym, co jednak nie uprawnia do zbiorowego wydania mów jednej osoby;
 3) przytaczać w dziełach, stanowiących samoistną całość, dla wyjaśnienia lub nauczania małe ustępy z wykładów, mów, oraz innych utworów naukowych i literackich, a z drobnych utworów najwyżej trzy z jednego dzieła, ale dopiero, gdy prace te zostały już wydane w książkach; do antologji wolno robić zapożyczenia z cudzych utworów, czy to drukowanych w książkach, czy w czasopismach, ale dopiero po śmierci autorów, z których czerpie się urywki;
 4) podawać krótkie streszczenia utworów ogłoszonych lub wystawionych;
 5) rozpowszechniać wydane dzieło przez odnajmowanie egzemplarzy, wygłaszanie wykładów, recytacje, jeśli się te recytacje odbywają nie w celach zarobkowych, chyba że autor ich wyraźnie zabronił. Wydany utwór sceniczny wolno wystawiać, lecz nie w teatrze i nie w celach zarobkowych;
 6) użyć już wydanych drobnych urywków utworu poetyckiego lub drobnych utworów poetyckich, jako tekstu nowego utworu sztuki muzycznej.


Art. 14.

 W zakresie utworów muzycznych wolno, pod warunkami art. 17:
 1) przytaczać w dziełach naukowych i literackich, albo w podręcznikach małe ustępy kompozycyj muzycznych lub drobne utwory w całości, o ile prace te już zostały wydane;
 2) rozpowszechniać wydane dzieła muzyczne przez odnajmowanie egzemplarzy, przez wygłaszanie wykładów z produkcjami jedynie objaśniającemi, przez wykonywanie samego dzieła, jeżeli za to wykonywanie nie pobiera się opłaty lub nie ma ono na celu innej korzyści materjalnej, albo jeżeli wykonanie stanowi część składową obchodu narodowego lub też, jeżeli towarzystwo muzyczne urządza wykonanie dzieła wyłącznie dla swych członków. Nie wolno wszakże wykonywać dzieła scenicznego w teatrze.


Art. 15.

 W zakresie utworów rysunkowych, malarskich, graficznych, rzeźbiarskich, architektonicznych i fotograficznych wolno, pod warunkiem art. 17:
 1) wystawiać dzieła publicznie, lecz nie dla zysku;
 2) umieszczać reprodukcje w dziełach naukowych, w podręcznikach i katalogach muzealnych lub używać ich do objaśnienia wykładów, jeżeli utwory zostały wydane, albo wystawione są stale w taki sposób, że każdy może je oglądać;
 3) kopjować w świątyniach lub muzeach dzieła, nabyte dla nich bezpośrednio od twórcy, jednakże z zachowaniem przepisów, ustalonych przez właściwy zarząd;
 4) odtwarzać jakąkolwiek techniką artystyczną lub reprodukcyjną dzieła sztuki, wystawione stale na drogach publicznych, ulicach, placach lub w publicznych ogrodach, jednakże nie w tych samych rozmiarach i nie dla takiego samego użytku; o ile chodzi o dzieła architektoniczne, można odtwarzać tylko zewnętrzną fasadę, a gdy chodzi o świątynie i gmachy publiczne — także ich wnętrza;
 5) odtwarzać w rzeźbie utwory malarskie lub graficzne i odwrotnie;
 6) budować według wydanych planów, opisów, modeli i rysunków budowlanych, jeżeli twórca, wydając je, nie zastrzegł wyłącznie dla siebie prawa budowania;
 7) odtwarzać utwory fotograficzne, lecz nie w sposób fotograficzny lub do niego podobny.


Art. 16.

 Minister Wyznań Religijnych i Oświecenia Publicznego może ze względów wyższej użyteczności upoważnić do rozpowszechniania wydanego dzieła zapomocą środków radjofonicznych lub radjowizyjnych, choćby twórcy lub nabywcy ich praw nie udzielili swego zezwolenia.
 Orzeczenie Ministra Wyznań Religijnych i Oświecenia Publicznego zarazem ustala należne za to upoważnienie słuszne odszkodowanie i powinno być doręczone na piśmie twórcy lub nabywcy jego praw.
 Orzeczenie może być wykonane przez osobę upoważnioną dopiero po wypłaceniu odszkodowania lub złożeniu go do depozytu sądowego.
 W ciągu miesiąca od dnia doręczenia orzeczenia twórca lub nabywca jego praw mogą w drodze powództwa wniesionego do sądu okręgowego, właściwego ze względu na siedzibę osoby upoważnionej, żądać podwyższenia odszkodowania. Wniesienie powództwa nie wstrzymuje wykonania orzeczenia.


Art. 17.

 Opracowanie, rozpowszechnianie, zapożyczanie z cudzych utworów, przewidziane w art. 2 oraz art. 13 — 16, dozwolone są tylko pod warunkiem wyraźnego podania źródła opracowania, zapożyczenia, rozpowszechniania i wymienienia twórcy.
 Wolność zapożyczenia nie upoważnia do żadnych zmian. W utworach muzycznych dozwolone są tylko przeniesienia na inny ton, na inny głos lub instrument, w dziełach zaś rysunkowych i plastycznych zmiany wielkości, tudzież zmiany konieczne, wywołane sposobem odtworzenia.


Art. 18.

 Wolno każdemu skopjować lub w inny sposób odtworzyć cudzy utwór wyłącznie dla własnego użytku prywatnego. Przepis ten nie odnosi się do budowania według cudzego utworu architektonicznego.


Art. 19.

 Na wykonywanie praw autorskich do wszelkiego rodzaju portretów potrzeba zewolenia osoby portretowanej, jeżeli nie otrzymała od artysty zapłaty.
 Zezwolenia nie potrzeba:
 1) jeżeli chodzi o wizerunki osób powszechnie znanych, a nie było z ich strony zastrzeżenia przy portretowaniu;
 2) jeżeli wizerunki osób są tylko szczegółem obrazu, przedstawiającego obchód, zgromadzenie, krajobraz i t. p.


Art. 20.

 Na wykonywanie praw autorskich co do listów potrzeba zezwolenia osoby, do której listy były zwrócone, jeżeli przez takie wykonywanie nazwisko jej ma lub może być ujawnione. Po śmierci tej osoby potrzeba do lat trzydziestu od jej zgonu zezwolenia małżonka, jeżeli nie było rozłączenia od stołu i łoża; w braku małżonka — zezwolenia rodziców; w braku rodziców — zezwolenia dzieci zmarłego; w braku tychże - zezwolenia rodzonych braci i sióstr.


R o z d z i a ł   III.
CZAS TRWANIA PRAWA AUTORSKIEGO.
Art. 21.

 Prawo autorskie gaśnie w pięćdziesiąt lat od śmierci twórcy; przy dziełach łącznych w pięćdziesiąt lat od śmierci tego twórcy, który innych przeżył.
 Względem utworów, niewydanych za życia (pośmiertnych), prawo autorskie gaśnie w pięćdziesiąt lat od jego śmierci. W razie wydania takiego utworu w ostatniem dziesięcioleciu, czas trwania prawa autorskiego przedłuża się o lat dziesięć.
 Prawo autorskie, które powstało na rzecz osób prawnych, gaśnie w pięćdziesiąt lat od czasu wydania utworu lub innego podania go do wiadomości publicznej. Ten sam termin stosuje się do anonimów i pseudonimów, jeżeli twórca przed wygaśnięciem prawa nie ujawnił publicznie swego autorstwa.
 Prawo autorskie do dzieł fotograficznych, lub otrzymanych w podobny do fotografji sposób gaśnie w dziesięć lat od zdjęcia fotografji; do utworów kinematograficznych — w dwadzieścia lat od sporządzenia filmu; do przeróbek utworów muzycznych na przyrządy mechaniczne — w dwadzieścia lat od dokonania przeróbki. Prawo autorskie do serji zdjęć fotograficznych, mającej znaczenie artystyczne lub naukowe, gaśnie w pięćdziesiąt lat od śmierci wydawcy.


Art. 22.

 Jeżeli utwór ukazuje się w częściach odrębnych (tomach, zeszytach i t. p.), to dla każdej części termin liczy się osobno; jeżeli jednak części nie są co do treści odrębnemi dziełami, czas trwania liczy się od wydania części ostatniej.


Art. 23.

 Czas trwania prawa autorskiego liczy się latami, począwszy od 1 stycznia tego roku, który nastąpił po śmieci twórcy, po prawowitem wydaniu lub innem zdarzeniu, oznaczonem w art. 21 i 22.


R o z d z i a ł   IV.
PRZEJŚCIE PRAW AUTORSKICH.
Postanowienia ogólne.
Art. 24.

 Prawo autorskie można przenosić na inne osoby przez czynności prawne między żyjącymi lub na przypadek śmieci; w braku rozporządzenia ostatniej woli, prawo to przechodzi na dziedziców ustawowych. Umowy, dotyczące przeniesienia prawa autorskiego, winny być pismem stwierdzone.


Art. 25.

 Prawo autorskie, dopóki służy twórcy, nie może być przedmiotem egzekucji z powodu roszczeń pieniężnych, jeżeli sprzeciwi się temu twórca. Po śmierci twórcy, jeżeli prawo autorskie służy dziedzicom, a dzieło nie zostało wydane, mogą się sprzeciwić egzekucji: małżonek twórcy, jeżeli nie było rozłączenia od stołu i łoża; w braku małżonka — rodzice; w braku rodziców — dzieci zmarłego; w braku dzieci — jego rodzeni bracia i siostry. Jednakże osoby te rozstrzygają tylko o tyle, o ile niema dostatecznych wskazówek co do woli twórcy w sprawie wydania dzieła.
 Ograniczenia powyższe nie obowiązują, jeżeli przedmiotem egzekucji jest prawo autorskie do utworów fotograficznych, lub otrzymywanych w podobny do fotografji sposób, do utworów kinematograficznych i do przeróbek muzycznych na instrumenty mechaniczne.


Art. 26.

 Stereotypy, płyty, kamienie, formy, oraz inne przyrządy, należące do uprawnionego i służące jedynie do wykonywania jego prawa autorskiego, stanowią przynależność tego prawa.


Art. 27.

 Prawa i obowiązki stron oceniać należy według umowy; o ile w niej brak wskazówek — według przepisów ustawy niniejszej; w braku tychże — według odpowiednich postanowień prawa handlowego i cywilnego.


Art. 28.

 Twórca, odstępując dzieło sztuki na własność, nie wyzbywa się przez to prawa autorskiego; jednak nabywca nie ma obowiązku dopuszczać twórcy do kopjowania, odtwarzania lub wystawiania dzieła.


Art. 29.

 Jeżeli przy sprzedaży oryginalnego dzieła sztuki plastycznej sprzedawca uzyskuje cenę przewyższającą więcej niż o połowę cenę nabycia, twórca i jego spadkobiercy przez cały czas trwania prawa autorskiego mają prawo do dwudziestoprocentowego udziału we wspomnianej przewyżce ponad połowę. Zrzeczenie się zgóry tego prawa ze strony twórcy lub jego spadkobierców nie ma skutków prawnych.
 Za zwrot udziału w przewyżce odpowiada przed uprawnionym sprzedawca.
 Udowodnienie ceny kupna i następnej ceny sprzedaży jest obowiązkiem uprawnionego.


Art. 30.

 Pomimo przeniesienia prawa autorskiego na inną osobę, twórca zachowuje swe prawa osobiste.


Art. 31.

 Następcy prawnemu, choćby nawet nabył wszelkie prawa autorskie, nie wolno czynić w utworze zmian, z wyjątkiem wywołanych oczywistą koniecznością, których twórca nie miałby słusznej podstawy zabronić.


Art. 32.

 Pomimo przeniesienia prawa autorskiego twórca nie traci wyłącznego prawa zezwalania na wykonywanie autorskich praw zależnych (art. 2), jeżeli nie umówiono się inaczej.


Art. 33.

 Każda strona może po czterech latach w każdym czasie wypowiedzieć na rok zgóry umowę, w której twórca obowiązuje się na czas powyżej lat pięciu oddawać drugiej stronie swe przyszłe utwory lub pewien ich rodzaj, albo też stale dla niej pewnym zakresie twórczości pracować.
 Zrzeczenie się tego prawa ze strony twórcy nie ma mocy prawnej.


Art. 34.

 Każda strona może przez umotywowane oświadczenie rozwiązać umowę o stworzenie dzieła aż do oddania tego dzieła, jeżeli po zawarciu umowy zaszły nieprzewidziane zdarzenia, będące słuszną przyczyną rozwiązania, jak to: choroba twórcy, wyłączająca na czas dłuższy wykonanie dzieła, okoliczności, zniewalające twórcę ze względu na jego istotne interesy duchowe do zaniechania dzieła, niewypłacalność zamawiającego i t. p. Przepis ten nie uchyla roszczeń z tytułu niesłusznego zbogacenia się, roszczeń o zwrot nakładów i odszkodowania.


Umowa o nakład.
Art. 35.

 Przez umowę o nakład (umowę wydawniczą) nakładca nabywa wyłączne prawo do wydania utworu piśmienniczego lub artystycznego i obowiązuje się uskutecznić je w stosownej formie, oraz użyć odpowiednich środków celem rozpowszechnienia wydawnictwa, przyczem dbać powinien o związane z wydawnictwem duchowe i materjalne interesy twórcy.


Art. 36.

 Nakładca nie może bez zezwolenia twórcy przenosić swych praw na inne osoby, chyba że je przenosi razem z przedsiębiorstwem, lecz i w tym przypadku twórca ma prawo zabronić przeniesienia w stosunku do swego nie wydanego dotąd dzieła, jeżeliby wyszły najaw fakty, wobec których wydawnictwo zaszkodziłoby poważnie dobrej sławie twórcy.
 Uważa się, że twórca udzielił zezwolenia, jeżeli nie sprzeciwił się przeniesieniu w ciągu dwóch miesięcy od otrzymania zawiadomienia o zamierzonem przeniesieniu.
 Twórca, zabraniający nakładcy przeniesienia praw na inną osobę wraz przedsiębiorstwem, obowiązany jest zwrócić otrzymane już za swój utwór wynagrodzenie.


Art. 37.

 Twórca obowiązany jest dostarczyć nakładcy całe dzieło, lub część, przeznaczoną do odrębnego wydania, bez zwłoki i w stanie odpowiednim; nakładca winien również bez zwłoki przystąpić do prac nad wydaniem i ukończyć je w należytym czasie.


Art. 38.

 Jeżeli twórca nie dostarczył nakładcy dzieła w czasie właściwym, nakładca może wyznaczyć mu odpowiadający okolicznościom czas dodatkowy z zagrożeniem rozwiązania umowy, a po bezskutecznym jego upływie rozwiązać umowę. Twórca może również rozwiązać umowę, jeżeli nakładca mimo oznaczenia mu czasu dodatkowego, okolicznościom odpowiadającego, z zagrożeniem rozwiązania umowy, nie podejmuje prac nad wydaniem dzieła. Co do roszczeń wzajemnych obowiązują ogólne przepisy prawa; jednak w razie niedostarczenia dzieła przez twórcę, nakładca może go skarżyć tylko o odszkodowanie, nie zaś o dokonanie dzieła.


Art. 39.

 Nakładca może uwolnić się od obowiązku wydania, płacąc umówione wynagrodzenie i zwracając dzieło, jeżeli strony inaczej nie postanowiły, jednak z prawa tego nakładca nie może korzystać po upływie sześciu miesięcy od dostarczenia mu dzieła, chyba że zaszły lub wyszły najaw fakty, wobec których wydawnictwo zaszkodziłoby poważnie dobru publicznemu, albo dobrej sławie nakładcy.


Art. 40.

 Przepisy art. 37 i 38 stosuje się także, gdy nakładca nabył od autora prawo do kilku wydań.
 Przed przystąpieniem do nowego wydania nakładca obowiązany jest dać autorowi możność poczynienia zmian w utworze. Jednak autorowi wolno poczynić tylko takie zmiany, którym nakładca sprzeciwić się nie miałby słusznej podstawy.


Art. 41.

 W braku odpowiedniej wskazówki w umowie wysokość wynagrodzenia oznacza się według zasad słuszności.
 W braku umowy o termin płatności wynagrodzenie płatne jest przy oddaniu dzieła nakładcy.
 W razie umowy o wynagrodzenie procentowe od całego nakładu procent oblicza się od ceny, po której egzemplarze sprzedaje się publiczności, a należność ma być zapłacona natychmiast po ukończeniu druku.
 Jeżeli wynagrodzenie zależy od liczby sprzedanych egzemplarzy, wydawca obowiązany jest co trzy miesiące przedstawiać twórcy rachunki, pozwalać mu przytem, względnie osobie, przezeń upełnomocnionej, przeglądać odpowiednie pozycje w księgach i fakturach i wypłacać przypadającą należność.
Art. 42.

 Nakładca ponosi koszty korekty.
 Twórca ma prawo żądać przesyłania sobie do przejrzenia korekt, wolnych od błędów drukarskich. Za poprawienie tych korekt nie należy się mu osobne wynagrodzenie.
 Twórca ponosi koszty zmian, dokonanych w dziele po rozpoczęciu pracy wydawniczej, jeżeli przekraczają zwykłą miarę, a nie są niezbędnem następstwem faktów, które zaszły niezależnie od twórcy po rozpoczęciu pracy wydawniczej.


Art. 43.

 W braku umowy co do liczby wydań i egzemplarzy nakładca ma prawo do jednego wydania najwyżej 2.000 egzemplarzy, a w 1.000 egzemplarzy, jeżeli chodzi o wydanie nut zwykłych.
 Twórcy należy się bezpłatnie po jednym egzemplarzu od każdej setki, jednak najwyżej 100. Nie wlicza się ich do powyższej liczby 2.000, względnie 1.000 egzemplarzy. Przy dziełach zbiorowych nakładca może zastąpić egzemplarze dzieła odbitkami danego przyczynku.
 Przepisy drugiego ustępu niniejszego artykułu nie dotyczą czasopism.


Art. 44.

 Poza liczbą egzemplarzy, określoną w art. 43, nakładca ma prawo celem dopełnienia powinności, przewidzianych w art. 35, zamówić nadwyżkę 100 egzemplarzy, a nadto celem pokrycia braków dalszą nadwyżkę po dwa egzemplarze od setki przy nakładach do 3.000 egzemplarzy, przy większych zaś nakładach po jednym jeszcze egzemplarzu od każdej dalszej setki.


Art. 45.

 Twórca, względnie osoba przezeń upełnomocniona, ma prawo sprawdzać w drukarni, ile drukuje się egzemplarzy dzieła i w tym celu wejrzeć w księgi zamówień, w faktury u nakładcy i w drukarni.


Art. 46.

 Cenę sprzedażną oznacza nakładca i zawiadamia o niej twórcę. Na podwyższenie ceny musi uzyskać zgodę twórcy, chyba że nie przekracza ona wzrostu kosztów takiego samego wydawnictwa. Jeżeli wynagrodzenie umówione zostało w odsetkach, to od podwyżki należy się twórcy umówiony procent za pozostałe do sprzedaży egzemplarze.


Art. 47.

 Twórca ma prawo przystąpić do nowego wydania niezwłocznie po rozprzedaniu poprzedniego.
 W każdej chwili służy mu prawo wykupienia od nakładcy niesprzedanych egzemplarzy po cenie, po jakiej nakładca sprzedaje je księgarzom.
 Bez względu na pozostały zapas egzemplarzy twórca może przystąpić do nowego wydania dzieła po upływie lat pięciu od ukazania się poprzedniego, a przy podręcznikach szkolnych i dziełach naukowych po upływie lat dziesięciu.


Art. 48.

 W wydaniu zbiorowem swych dzieł twórca może umieścić również takie utwory, co do których prawo nakładu odstąpił osobom innym, jeżeli od czasu ich ukazania się upłynęło lat pięć; jednakże nie może sprzedawać ich oddzielnie, chyba że ma prawo do ich wydania na podstawie artykułu poprzedniego.
 Prawo nakładcy do wydania zbiorowego utworów jednego autora nie obejmuje prawa wydawania lub sprzedawania oddzielnie poszczególnych utworów.


Inne umowy o rozpowszechnianie utworów.
Art. 49.

 Przy umowach o publiczne wystawienie dzieła scenicznego, lub o publiczne wykonanie dzieła muzycznego stosuje się z odpowiednimi zmianami artykuły 35 do 38 i 40.
 Przepisy art. 41 stosuje się z tą zmianą, że wynagrodzenie autorskie płatne jest natychmiast po zawarciu umowy, a jeżeli twórca ma dostarczyć rękopis, natychmiast po dostarczeniu go przedsiębiorcy; jeżeli zaś wynagrodzenie oblicza się w stosunku do zysku (tantjema), należy je wypłacać po każdem zamknięciu kasowem.


Art. 50.

 Twórca może natychmiast rozwiązać umowę, jeżeli przedsiębiorca wystawia utwór w rażąco nieodpowiedniej formie, lub zupełnie nieodpowiedniemi siłami, albo wprowadza zmiany, którym sprzeciwiać się twórca miałby słuszną podstawę.


Art. 51.

 Kto nabył za wynagrodzeniem niewydane plany architektoniczne, nabywa prawo zastosowania ich tylko w jednej budowli.


Art. 52.

 Jeżeli twórca zezwala na przeniesienie swego utworu na instrumenty mechaniczne, to, w braku umowy odmiennej, uważa się, że zezwolenie nie obejmuje uprawnienia do publicznego wykonywania dzieła za zapłatą lub dla jakiejkolwiek korzyści materjalnej.
 Przepis powyższy stosuje się także w przypadkach, gdy przy przeniesieniu dzieła została dokonana jego przeróbka, uzasadniająca ze względu na twórczą działalność powstanie zależnego prawa autorskiego.


Art. 53.

 Zezwolenie na przeniesienie działa na film kinematograficzny obejmuje, w braku umowy odmiennej, uprawnienie do publicznego wykonywania filmu.


Art. 54.

 Zezwolenie na publiczne wykonywanie dzieła nie obejmuje, w braku umowy odmiennej, zezwolenia na rozpowszechnianie dzieła zapomocą radjofonji lub radjowizji.
 Posiadacze głośników lub innych podobnych urządzeń, nawet umieszczonych w miejscu publicznem, mają prawo, bez odrębnego wynagrodzenia dla twórców, używać wspomnianych urządzeń przy odbiorze dzieła, rozpowszechnianego zapomocą radjofonji lub radjowizji.
R o z d z i a ł   V.
UMOWA AGENCYJNA.
Art. 55.

 Umowa agencyjna uprawnia i zobowiązuje agenta do udzielenia we własnym imieniu, ale na rachunek twórcy, licencji na przedstawienia utworów scenicznych i do wykonywania dzieł muzycznych, tudzież do ścigania sądownie w imieniu twórcy bezprawnych przedstawień, względnie wykonywań tych utworów.


Art. 56.

 Prawa i obowiązki, wynikające z umowy twórcy z agentem, należy oceniać na podstawie art. 27.
 Umowa ta nie uprawnia agenta do zawierania umów o wydanie dzieła.


Art. 57.

 Agent obowiązany jest donosić twórcy natychmiast o każdej umowie licencyjnej, na jego rachunek zawartej, składać mu rachunki i wypłacać co trzy miesiące wynagrodzenie autorskie lub tantjemy, po potrąceniu prowizji, która w braku umowy wynosi dziesięć od sta.


Art. 58.

 Śmierć agenta rozwiązuje umowę.


R o z d z i a ł   VI.
OCHRONA PRAWNA.
Skargi z powodu naruszenia praw autorskich.
Art. 59.

 Twórca (lub jego prawny następca) może żądać od wkraczającego bezprawnie w jego prawa, by zaniechał naruszenia, wydał to, czem się zbogacił, a w razie winy wynagrodził wszelką szkodę.


Art. 60.

 Bezprawnie sporządzone egzemplarze lub ich części oraz przyrządy, służące do wydawnictwa, jak: klisze, stereotypy, kamienie, płyty i t. p., należące do pozwanego, mają być na wniosek pokrzywdzonego przyznane mu na poczet roszczeń pieniężnych, albo pozostawione u właściciela w stanie niezdatnym do użytku. Nie można jednak niszczyć dzieł sztuki.
 Przy dziełach budownictwa nie można wstrzymać rozpoczętej budowy. Pokrzywdzony ma jednak prawo do wynagrodzenia (honorarjum) według sprawiedliwego uznania, niezależnie od roszczeń z tytułu niesłusznego zbogacenia się i roszczeń o odszkodowanie.


Art. 61.

 Od osoby, która, nie wkraczając w prawo autorskie, wyrządza z winy swej szkodę w jego przedmiocie, pokrzywdzony żądać może wynagrodzenia zrządzonej szkody.


Skargi z powodu naruszenia praw osobistych.
Art. 62.

 Twórca, któremu wyrządzono szkodę w zakresie jego osobistego stosunku do dzieła, może, — chociażby prawo autorskie wcale nie istniało, zgasło, przeszło na inne osoby, albo było bezskuteczne według postanowień art. 13 do 16 — żądać niezależnie od roszczeń z art. 59 do 61 zaniechania czynów krzywdzących i usunięcia ich skutków, w szczególności publicznego odwołania lub innej deklaracji publicznej, ogłoszenia wyroku w czasopismach i innych środków zadośćuczynienia. Jeżeli czyn był popełniony rozmyślnie, sąd na wniosek pokrzywdzonego może mu oprócz odszkodowania przyznać za poniesione przykrości i inne osobiste uszczerbki odpowiednią kwotę, którą oznaczy stosownie do zachodzących okoliczności według swobodnego uznania (pokutne).
 Taką krzywdą osobistą jest: gdy ktoś przywłaszcza sobie autorstwo, nazwisko twórcy lub pseudonim; gdy nie podaje w swym utworze autora lub źródła, z którego zaczerpnął treść lub wyimki, przez co może powstać błędne mniemanie co do autorstwa, albo podaje fałszywie autora lub źródło; gdy publikuje dzieło, do publikacji przez twórcę nie przeznaczone; gdy wprowadza w publikacji zmiany, dodatki, skrócenia, które treść wykrzywiają, lub uwłaczają godności i wartości dzieła; gdy wydaje dzieło w rażąco nieodpowiedni sposób; gdy czyni zmiany w oryginale dzieła, gdy oryginał dzieła sztuki oznacza nazwiskiem twórcy wbrew jego woli, lub w inny sposób wbrew jego woli ujawnia autorstwo; gdy w krytyce obniża wartość dzieła przez świadomie fałszywe przedstawienie faktów i t. p.


Art. 63.

 Po śmierci twórcy powołani są do wniesienia skargi z art. 62, jeżeli twórca nie wyraził innej woli, jego małżonek, rodzice, zstępni oraz rodzeni bracia i siostry zmarłego. Osoby te jednak nie mają prawa do pokutnego. W razie wniesienia skargi przez jedną z nich inne nie mogą wytaczać skargi osobnej, lecz mogą tylko przystąpić do już wytoczonego sporu.
 Niezależnie od osób, wymienionych w ustępie poprzedzającym, po śmierci twórcy może wnieść pozew samodzielny z art. 62 Prokuratorja Generalna Rzeczypospolitej Polskiej w interesie publicznym, na polecenie Ministerstwa Wyznań Religijnych i Oświecenia Publicznego. Pozew może również obejmować żądanie pokutnego.


Art. 64.

 Przepis art. 62 stosuje się odpowiednio do rozpowszechnienia portretu bez pozwolenia portretowanego; przepisy art. 62 i 63 — do naruszanie praw osobistych przez wydanie listów bez pozwolenia, wymaganego art. 20.


Postępowanie zabezpieczające.
Art. 65.

 W sprawach o roszczenia z zakresu prawa autorskiego ma zastosowanie art. 859 kodeksu postępowania cywilnego, choćby chodziło o zabezpieczenie roszczeń pieniężnych.


Art. 66.

 Jeżeli naruszenie prawa autorskiego nastąpiło w miejscu, w którem niema siedziby sądu okręgowego, zarządzenia tymczasowe, dotyczące zabezpieczenia przed wytoczeniem powództwa, może wydać także sąd grodzki, który byłby miejscowo właściwy do rozpoznania sprawy.
 Art. 840 kodeksu postępowania cywilnego stosuje się odpowiednio.


Art. 67.

 Postanowienie sądu grodzkiego o zabezpieczeniu powództwa nie ulega zaskarżeniu.
 Sąd okręgowy, do którego wpłynął pozew w terminie wyznaczonym (art. 840 k. p. c.), zarządzi na wniosek pozwanego natychmiast rozprawę celem powzięcia postanowienia o utrzymaniu w mocy lub uchyleniu zarządzenia tymczasowego.


Postanowienia karne.
Art. 68.

 Kto wbrew przepisom ustawy niniejszej wkracza w wyłączne prawa twórcy lub jego następcy prawnego, podlega karze aresztu do sześciu miesięcy lub grzywny do dziesięciu tysięcy złotych albo obu tym karom łącznie.
 Sporządzający kopję z cudzego utworu sztuki plastycznej podlega karze tylko wtedy, gdy trudni się takiem kopjowaniem zarobkowo.


Art. 69.

 Nakładca, który bez wiedzy twórcy zamawia, i sporządzający nakład, który bez wiedzy twórcy wytwarza większą liczbę egzemplarzy utworu, aniżeli mu dozwolono, podlega karze aresztu do dwóch lat lub grzywny do pięćdziesięciu tysięcy złotych, albo obu tym karom łącznie.


Art. 70.

 Kto przywłaszcza sobie cudze autorstwo, podlega karze aresztu do jednego roku lub grzywny do dziesięciu tysięcy złotych, albo obu tym karom łącznie.


Art. 71.

 Ściganie czynów z art. 68 i 70 może nastąpić tylko z oskarżenia prywatnego.


Art. 72.

 Do ścigania z oskarżenia prywatnego są uprawnione te osoby, którym w danym przypadku służy prawo do skargi cywilnej. Uprawnienie to nie przysługuje Prokuratorji Generalnej Rzeczypospolitej Polskiej w przypadkach, gdy występuje na zasadzie art. 63 ust. 2.


Art. 73.

 Sąd może na wniosek pokrzywdzonego zarządzić ogłoszenie wyroku na koszt skazanego.


Art. 74.

 W sprawach o przestępstwa z art. 68, 69 i 70 orzekają sądy okręgowe.


R o z d z i a ł   VII.
POSTANOWIENIA PRZEJŚCIOWE I KOŃCOWE.
Art. 75.

 Ustawa niniejsza stosuje się także do praw autorskich, istniejących w dniu jej wejścia w życie. Przez to jednakże nie skraca się czasu trwania praw, przez dotychczasowe przepisy określonego, a przedłuża się go jedynie, gdy prawem autorskiem rozporządza jeszcze twórca lub jego spadkobierca.


Art. 76.

 Umowy, dotyczące przejścia prawa autorskiego, ocenia się według przepisów, które obowiązywały w czasie zawarcia umowy.


Art. 77.

 Przedruki, reprodukcje, budowy i przeróbki dla instrumentów mechanicznych, nie zabronione przez dotychczasowe obowiązujące przepisy, a rozpoczęte, zanim ustawa niniejsza uzyskała moc obowiązującą, wolno ukończyć i rozpowszechniać, chociażby podpadały pod zakazy niniejszej ustawy.


Art. 78.

 Ustawa niniejsza wchodzi w życie w trzydzieści dni po jej ogłoszeniu.


Art. 79.

 Wykonanie niniejszej ustawy porucza się Ministrowi Wyznań Religijnych i Oświecenia Publicznego oraz Ministrowi Sprawiedliwości.


Ten tekst nie jest objęty majątkowymi prawami autorskimi lub prawa te wygasły. Jest zatem w domenie publicznej. Więcej informacji na stronie dyskusji.
Kategorie:

    Prawo historycznePublic domainPrawo autorskieDziennik Ustaw


	Ustawa o prawie autorskiem 	

    Tę stronę ostatnio edytowano 16 sie 2019, 19:06.
    Tekst udostępniany na licencji Creative Commons: uznanie autorstwa, na tych samych warunkach, z możliwością obowiązywania dodatkowych ograniczeń. Zobacz szczegółowe informacje o warunkach korzystania.

    Privacy policy
    O Wikiźródłach
    Informacje prawne
    Powszechne Zasady Postępowania
    Dla deweloperów
    Statystyki
    Komunikat na temat ciasteczek
    Wersja mobilna

    Wikimedia Foundation
    Powered by MediaWiki




%%% Local Variables: 
%%% coding: utf-8-unix
%%% mode: latex
%%% TeX-master: t
%%% TeX-PDF-mode: t
%%% TeX-engine: xetex
%%% TeX-command-extra-options: "-shell-escape"
%%% End: 
